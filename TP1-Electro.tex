
\documentclass[]{article}

\usepackage[]{graphicx}   % para manejar graficos

\usepackage{caption}

\usepackage{enumerate}    % para hacer listas numeradas

\usepackage{amsmath}        % no se..

\usepackage{amsfonts}     % no se..

\usepackage{authblk}    % para definir las afiliaciones de cada autor

\usepackage{layout}     % no se..

\usepackage{biblatex}     % para manejar la bibliografia / referencias

\usepackage{lipsum}     % para generar texto random

\usepackage{multicol}   % para usar dos columnas

\usepackage{palatino}   % para que la fuente sea palatino

\usepackage[utf8]{inputenc} % para poder usar tildes

\usepackage[spanish]{babel} % para escribir en español

\addto\captionsspanish{
\def\tablename{Tabla}
}

\usepackage[sc,big,raggedright,bf]{titlesec} % para definir el formato del
%                                              header de cada seccion.

\usepackage[font=small]{caption} % para que la fuente de un epigrafe no tenga el
%                                  mismo tamaño que el cuerpo del texto

\usepackage{geometry}
 \geometry{
 a4paper,
 textwidth={17cm},
 textheight={23cm},
 left={2cm},
 top={2.5cm},
 }

\setlength{\columnsep}{1cm} % para que la separacion entre columnas sea de 1 cm

\graphicspath {{imagenes/}}

\defbibheading{bibliography}{\section{\refname}} % para que bibtex no imponga su
 % header cuando uso \printbibliography, y que se use el de babel

\addbibresource{bibliografia.bib} % para importar el archivo .bib

\title{\textbf{\LARGE{\textsf{MEDICIÓN DE RESPUESTA EN FRECUENCIA Y
 SENSIBILIDAD DE VARIOS MICROFONOS}}}}
 % defino el titulo del Paper

\date{} % lo pongo vacio para que no aparezca abajo del abstract

\usepackage{fancyhdr}

%%%%%%%%%%%%%%%%%%%%%%%%%%%%%%%%%%%%%%%%%%%%%%%%%%%%%%%%%%%%%%%%%%%%%%%%%%%%%%%%
% http://www-h.eng.cam.ac.uk/help/tpl/textprocessing/multicol_hint.html
\makeatletter           % esto lo uso para poder definir figuras
\newenvironment{tablehere}    % esto lo uso para poder definir figuras
  {\def\@captype{table}}    % esto lo uso para poder definir figuras

  {}              % esto lo uso para poder definir figuras
                  % esto lo uso para poder definir figuras
\newenvironment{figurehere}   % esto lo uso para poder definir figuras
  {\def\@captype{figure}}   % esto lo uso para poder definir figuras
  {\par\medskip}
  {}              % esto lo uso para poder definir figuras
\makeatother          % esto lo uso para poder definir figuras
%%%%%%%%%%%%%%%%%%%%%%%%%%%%%%%%%%%%%%%%%%%%%%%%%%%%%%%%%%%%%%%%%%%%%%%%%%%%%%%%

%%%%%%%%%%%%%%%%%%%%%%%%%%%%%%%%%%%%%%%%%%%%%%%%%%%%%%%%%%%%%%%%%%%%%%%%%%%%%%%%
%               ACA EMPIEZA EL DOCUMENTO                            %
%%%%%%%%%%%%%%%%%%%%%%%%%%%%%%%%%%%%%%%%%%%%%%%%%%%%%%%%%%%%%%%%%%%%%%%%%%%%%%%%


\begin{document} % empieza el documentoo


\renewcommand{\headrulewidth}{0pt} % para que no haya linea decorativa en el header.


\author[1]{Federico Feldsberg} % defino el autor
\affil[1]{Universidad Nacional de Tres De Febrero, Buenos Aires, Argentina \newline \texttt{fedefelds@hotmail.com}} % afiliacion del autor


\begin{minipage}[h]{\textwidth} % uso el entorno minipage para que el abstract este en la misma pagina que el titulo
    \maketitle
    \thispagestyle{fancy}
    \fancyhf{}
    \rhead{9 de Junio}
    \lhead{Electroacústica I}
    \cfoot{\thepage}

\end{minipage}


\begin{abstract}
\textit{En este artículo se describe una serie de mediciones llevadas a cabo
en el marco de la materia Electroacústica I. Se recurre a dos métodos
de medición distintos para caracterizar los microfonos Shure SM57,
Rode NT2000, Earthworks M50 y Beyerdynamic MM1}
\end{abstract}

\begin{multicols}{2}
\section{Introducción}
El presente trabajo tiene como objetivo describir los procesos de medición
llevados a cabo en el marco de la materia Electroacustica I, así como analizar
los resultados allí obtenidos desde el marco teorico de la misma.

Las mediciones están relacionadas con los conceptos de sensibilidad y respuesta
 en frecuencia de micrófonos. Se consideran dos micrófonos durante todo el
proceso: uno se utiliza como micrófono referencia, y el micrófono a medir, de
sensibilidad y respuesta en frecuencia desconocidas. Se utilizaron dos métodos
de medición: el método clásico de Davis y otro método moderono basado en la
función de transferencia.
\section{Método clásico}
El metodo clásico propuesto por Davis en \cite{davis2006sound} permite medir
la sensibilidad y la respuesta en frecuencia de un microfono dado.

Para este método se utiliza el siguiente instrumental:
\begin{itemize}
\item una consola mezcladora Behringer
\item un parlante de monitoreo KRK
\item un sonometro Svantek y su calibrador
\item un multimetro digital Uni-T
\item un osciloscopio digital Tektronix
\item un microfono de referencia Earthworks M50
\end{itemize}

\subsection{Medición de sensibilidad}
En primer lugar, se mide el piso de ruido en el lugar de medición para asegurarse
de que el mismo no afecte las demas mediciones. Se calibra el sonómetro con el
tono puro de 1 kHz que emite su calibrador, de manera que este último detecte un
nivel de presión sonora de referencia de 94 dBSPL, y calcule las compensaciones
necesarias. Ademas se le asigna un tiempo de integración lento.

Dicho método consiste en utilizar un microfono de referencia (Earthworks M50)
cuya sensibilidad es conocida. Debido a sus respuesta en frecuencia en respuesta
practicamente plana, es considerado un microfono ideal. Tanto el microfono de
referencia como el microfono a medir se colocan a una distancia de 15 cm del
parlante para poder asegurar el nivel de presion sonora deseado.

La figura \ref{fig:bloque_clasico} indica el arreglo instrumental utilizado en el
método clásico

\begin{figurehere}
 \centering
 \includegraphics[width=\linewidth]{blockdiag}
 \captionof{figure}{Arreglo instrumental del método clásico}
 \label{fig:bloque_clasico}
\end{figurehere}

El micrófono de referencia es expuesto a un tono puro de 1 kHz a un nivel de
presion sonora de 94 dB SPL. El nivel de presión sonora es controlado con el
sonómeto Svantek. Debido a que no poseemos instrumental lo suficientemente
preciso para medir tensiones en el orden de magnitud de los mV, debemos amplificar
la tension generada en los terminales del microfono mediante el uso de una
consola mezcladora. La forma de onda de la salida de dicha consola es monitoreada
por medio de un osciloscopio digital, a modo de hallar una ganancia tal que sea
medible por nuestro instrumental y aun asi no ocasione recortes de señal. Dicha
ganancia se fija para toda la medicion.

La sensibilidad de un microfono $S_0$ esta definida como la tension en los terminales
de salida del mismo, al estar expuesto a un nivel de presion sonora de 94 dB SPL.
Debido a que la sensibilidad de nuestro microfono de referencia es conocida,
podemos averiguar la ganancia de la consola si consideramos las ecuaciones
\ref{eq:veces} y \ref{eq:dB}.

\begin{equation}
  G_v=\frac{V_o}{V_i}
  \label{eq:veces}
\end{equation}

\begin{equation}
  G_{dB}= 20 \log \left(\frac{V_o}{V_i}\right)
  \label{eq:dB}
\end{equation}

Bajo las condiciones de esta medición $V_o$ es conocido y $V_i$=$S_0$, por lo que
la ganancia tanto en veces como en dB queda determinada por las ecuaciones
\ref{eq:g-veces} y \ref{eq:g-dB} respectivamente:

\begin{equation}
  G_v=\frac{V_o}{S}
  \label{eq:g-veces}
\end{equation}

\begin{equation}
  G_{dB}= 20 \log \left(\frac{V_o}{S}\right)
  \label{eq:g-dB}
\end{equation}

Luego se intercambia el microfono de referencia por el Shure SM57, cuya
sensibilidad desconocida es $S_1$. Dicho transductor es expuesto a un tono puro
de 1 kHz a 94 dB SPL y la tensión en la salida de la consola mezcladora es medido.
Por lo tanto, la sensibilidad del microfono bajo medicion esta dada por la ecuación

\ref{eq:sensibilidad}:
\begin{equation}
  S_1=\frac{V_o}{G_v}
  \label{eq:sensibilidad}
\end{equation}

\subsection{Medición de respuesta en frecuencia}
La medición de respuesta en frecuencia se basa en un arreglo instrumental
similar al usado en la medición de sensibilidad según el método clásico.
La única diferencia esta en el nivel de presión sonora de la señal utilizada: Debido
a que el monitor no es capaz de manejar un nivel de 94 dB SPL en altas
frecuencias, se uso 84 dB SPL.
En los ajustes del generador se regula la frecuencia del tono puro y al mismo
tiempo se mide la tensión en la salida de la consola mezcladora en 100 Hz, 500 Hz,
1 kHz y 10 kHz.

\section{Método moderno}
el metodo moderno propuesto se basa el empleo de una función de transferencia
\cite{smaart} y permite medir la respuesta en frecuencia de un microfono dado.

Para este método se utiliza el siguiente instrumental:

\begin{itemize}
\item un micrófono de referencia Earthworks M50
\item una interfaz USB
\item Software Smaart V 7.4
\item un parlante de monitoreo KRK
\item micrófonos Shure SM57 y Rode NT2000
\item micrófono Beyerdynamic MM1
\end{itemize}

En primer lugar, se mide el piso de ruido en el lugar de medición para asegurarse
de que el mismo no afecte las demás mediciones. Luego se coloca el micrófono de
referencia y el micrófono a medir frente al centro acústico del monitor.
Las cápsulas de ambos micrófonos deben estar lo mas cerca posible para que el
campo acústico captado por ambos sea lo mas parecido posible.

La figura \ref{fig:bloque_moderno} indica el arreglo instrumental del metodo
moderno:

\begin{figurehere}
 \centering
 \includegraphics[width=\linewidth]{blockdiag}
 \captionof{figure}{Arreglo instrumental del método moderno CAMBIAAAR}
 \label{fig:bloque_clasico}
\end{figurehere}

Ambos microfonos son expuestos a ruido rosa y las señales captadas por ambos
es procesada por el Software. Este ultimo compara ambas señales y bajo la
suposicion que micrófono de referencia Earthworks M50 es un microfono con
respuesta en frecuencia plana, permite obtener la respuesta en frecuencia del
microfono a medir.

Los microfonos medidos son : Shure SM57, Rodes NT2000, Earthworks M50 y
Beyerdynamic MM1. El Earthworks M50 y el Beyerdynamic MM1 fueron medidos en eje
y a $90^\circ$.

\section{Resultados}
En la siguiente seccion exponemos los resultados de las mediciones realizadas:
\subsection{Método clasico}

El calibrador del sonometro establece una correccion de 0.02 dB. En un intervalo
de 10 segundos, dicho sonometro indica un nivel de ruido equivalente de 75,3 dB
$L{eq}$. Considerando que $S_0$= $34$ mV/Pa , la sensibilidad del Shure SM57 medida es de
$1$ mV/Pa.

La tabla \ref{tab:frespsm57} presenta los resultados de la medicion de respuesta en
frecuencia del microfono Shure SM57. La tercer columna presenta los valores en dB
referidos a 1kHz.

\begin{tablehere}
\begin{center}
\begin{tabular}{|c|c|c|c|c|}
\hline
Frecuencia & Sensibilidad [mV] & Sensibilidad [dB] \\
\hline
100 Hz & 690    & -1,8 \\
500 Hz & 580    & -3,3 \\
1  kHz & 850    & 0    \\
10 kHz & 1150   & 2,6  \\
\hline
\end{tabular}
\caption{Valores obtenidos en la medicion de respueta en frecuencia del
Shure SM57}
\label{tab:frespsm57}
\end{center}
\end{tablehere}
La figura \ref{fig:frespsm57} es una representacion grafica de la informacion
presentada en la tabla \ref{tab:frespsm57}:

\begin{figurehere}
 \centering
 \includegraphics[width=\linewidth]{frespsm57}
 \captionof{figure}{Respuesta en frecuencia del SM57}
 \label{fig:frespsm57}
\end{figurehere}


% import matplotlib.pyplot as plt
% x=[100, 500, 1000, 10000]
% y=[-1.8, -3.3, 0, 2.6]
% plt.plot(x,y,color='black')
% plt.xscale('log')
% plt.axis([0,10000,-10,3])
% plt.title('Nivel de sensibilidad referido a 1 kHz', fontsize=18)
% plt.xlabel('Frecuencia [Hz]', fontsize=18)
% plt.ylabel('Sensibilidad [dB]', fontsize=18)
% plt.grid(True)



\subsection{Método moderno}
Las curvas de respuesta en frecuencia son exportados y graficados en Python,
mediante el uso de la libreria \textit{Matplotlib}
\begin{itemize}
  \item piso de ruido
  \item factor de calibracion
\end{itemize}
\section{Análisis de resultados}



\section{Conclusiónes}
NO PUEDE FALTAR MI TP NO PUEDE FALTAR MI TP
NO PUEDE FALTAR MI TP NO PUEDE FALTAR MI TP
Supuestamente el earthorks es omni, pero en altas frecuencias no tanto

\printbibliography

\end{multicols}

\end{document}
